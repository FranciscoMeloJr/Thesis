% Résumé du mémoire.
%
%   Le résumé est un bref exposé du sujet traité, des objectifs visés,
% des hypothèses émises, des méthodes expérimentales utilisées et de
% l'analyse des résultats obtenus. On y présente également les
% principales conclusions de la recherche ainsi que ses applications
% éventuelles. En général, un résumé ne dépasse pas quatre pages.
%
%   Le résumé doit donner une idée exacte du contenu du mémoire ou de la thèse. Ce ne
% peut pas être une simple énumération des parties du document, car il
% doit faire ressortir l'originalité de la recherche, son aspect
% créatif et sa contribution au développement de la technologie ou à
% l'avancement des connaissances en génie et en sciences appliquées.
% Un résumé ne doit jamais comporter de références ou de figures.
\chapter*{RÉSUMÉ}\thispagestyle{headings}
\addcontentsline{toc}{compteur}{RÉSUMÉ}

La performance est devenue une question cruciale sur le développement, le test et la maintenance des logiciels. Pour répondre à cette préoccupation, les développeurs et les testeurs utilisent plusieurs outils pour améliorer les performances ou suivre les bogues liés à la performance.\\
L'utilisation de méthodologies comparatives telles que Flame Graphs fournit un moyen formel de vérifier les causes des régressions et des problèmes de performance. L'outil de comparaison fournit des informations pour l'analyse qui peuvent être utilisées pour les améliorer par un mécanisme de profilage profond, comparant habituellement une donnée normale avec un profil de profil anormal.\\
D'autre part, le mécanisme de traçage est un mécanisme de tendance visant à enregistrer des événements dans le système et à réduire les frais généraux de son utilisation. Le registre de cette information peut être utilisé pour fournir aux développeurs des données pour l'analyse de performance. Cependant, la quantité de données fournies et les connaissances requises à comprendre peuvent constituer un défi pour les méthodes et les outils d'analyse actuels.
La combinaison des deux méthodologies, un mécanisme comparatif de profilage et un système de traçabilité peu élevé peut permettre d'évaluer les causes des problèmes répondant également à des exigences de performance strictes en même temps. La prochaine étape consiste à utiliser ces données pour développer des méthodes d'analyse des causes profondes et d'identification des goulets d'étranglement.\\
L'objectif de ce projet de recherche est d'automatiser le processus d'analyse des traces et d'identifier automatiquement les différences entre les groupes d'exécutions. La solution présentée souligne les différences dans les groupes présentant une cause possible de cette différence, l'utilisateur peut alors bénéficier de cette revendication pour améliorer les exécutions.\\
Nous présentons une série de techniques automatisées qui peuvent être utilisées pour trouver les causes profondes des variations de performance et nécessitant des interférences mineures ou non humaines. L'approche principale est capable d'indiquer la performance en utilisant une méthodologie de regroupement comparative sur les exécutions et a été appliquée sur des cas d'utilisation réelle. La solution proposée a été mise en œuvre sur un cadre d'analyse pour aider les développeurs à résoudre des problèmes similaires avec un outil différentiel de flamme.\\
À notre connaissance, il s'agit de la première tentative de corréler les mécanismes de regroupement automatique avec l'analyse des causes racines à l'aide des données de suivi.\\
Dans ce projet, la plupart des données utilisées pour les évaluations et les expériences ont été effectuées dans le système d'exploitation Linux et ont été menées à l'aide de Linux Trace Toolkit Next Generation (LTTng) qui est un outil très flexible avec de faibles coûts généraux.