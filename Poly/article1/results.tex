% Present the results for your research questions. For each one, provide a short motivation (why is this question important to answer?), brief overview of analyses used (of those discussed in \autoref{sec:approach} and your actual results. For each individual result of a research question, put a first sentence in bold, then use the rest of the paragraph and possibly follow-up paragraphs to explain the result. Do this for each result of a question, and all questions.

This section presents a discussion related with the research questions presented on the introduction, as well as, the solution presented. The results of the use cases clarifies the research questions delimited on the introduction and conduct new insights on this performance part.

\begin{itemize}

\item RQ 1: How can we build an efficient and flexible model for performance comparison?\\
    Using the enhanced calling context tree it is possible to aggregate performance metrics and compare executions efficiently. The CCT can be built with or without source code modification. The possibility to build it without source code brings the opportunity to analyze the dynamic structure and an accurate behaviour can be used.\\
    
\item RQ 2: Is it possible to automate the performance analysis?\\
    Our work demonstrated the possibility to use non-supervised machine learning methods to compare and find performance problems, specifically using a comparative heuristic technique. The approach used is to use a comparative clustering mechanism, which can efficiently separate the data for a realistic comparison.\\
    
\item RQ 3: How accurate was the obtained results?\\
    The methodology used was able to find association between groups of metrics and consequently associate the cause of elapsed time variations. However this approach can have type I or type II errors - false positives and false negatives.
    
\end{itemize}

% The solution presented was able to automate the process of clustering by using a deterministic heuristic comparison.
